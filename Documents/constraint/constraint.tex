\documentclass[12pt]{extarticle}
\usepackage[top=0.5in, bottom=0.8in, left=0.5in, right=0.5in]{geometry}

\usepackage{amsmath,amssymb,amsfonts,amsthm}            
\usepackage[english]{babel}
\usepackage[T1]{fontenc}
\usepackage[latin1]{inputenc}
\usepackage{bm}
\usepackage{verbatim}
\usepackage[all]{xy}
\usepackage[pdftex]{hyperref}
\usepackage{multirow}
\hypersetup{colorlinks=false, linkcolor=blue, pdffitwindow=true, pdftitle={Note of Asymptotic property}}
\DeclareMathOperator*{\argmax}{argmax}
\DeclareMathOperator*{\argmin}{argmin}

\title{Note on the constraint condition}

\begin{document}
\maketitle

We can not do anything to $k_{nf} > 0, k_{pf} > 0$ but we can change the constraint condition $(9)$ in Theorem 3. \\

There are three inequalities in constraint condition $9$ \\

1. $\lambda > 0$ \\

There are two ways to handle this condition. \\
The first one is to mention the implied condition $E[Y_j] > 0$ which will obviously replace the condition about $\lambda$ because $\sigma(x) \geq 0$. \\
The second one is that we just change the condition as $\lambda \geq 0$ because we know that it is safe if the sample data is well defined. \\

2. $\mu > 0$ \\

Thanks to the good even property of function $\sigma_0$, there is no problem if $\mu \neq 0$. Actually, under the implied condition that the intensity $Y$ depends on the position $X$, we can easily get that $\mu = 0$ is impossible which means we can remove this condition. \\

3. $\Lambda^{*}(\mu, \lambda) = \mu R_{tot} - \lambda \|\sigma_0\|_1 > 0$ \\
This condition transfers and simplifies the condition just before Theorem 3. We can only transfer but not simplify the origin condition. Then the constraint condition of Theorem 3 will be like this:

\begin{equation}
  \left\{
  \begin{array}{l}
    \Lambda^{*}(\mu, \lambda) = \lambda^{\alpha - 1} - \frac{1}{\alpha}\left(\frac{\alpha - 1}{\alpha}\mu\frac{R_{tot}}{\|\sigma_0\|_1}\right)^{\alpha - 1} - \frac{1}{R_{tot}}\lambda^{\alpha}\frac{1}{\mu}\|\sigma_0\|_1 \leq 1 \\
    \lambda \geq 0
  \end{array}\right.
\end{equation}

\end{document}
